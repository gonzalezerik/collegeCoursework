\documentclass{article}
\usepackage[utf8]{inputenc}
\usepackage{amsmath}
\title{CSE 15 Homework 1 : Mathematical Typesetting}
\author{Erik Gonzalez}
\date{February 7 2020}

\begin{document}

\maketitle

\section*{Exercise 1}
Solve $(x-2)(x+4) = 7$\\
\begin{align*}
    x^2 +4x - 2x - 8 - 7 = 0\\
    x^2 + 2x - 15 = 0\\
    (x - 3)(x + 5) = 0\\
\end{align*}

Therefore $x = 3$ or $x = -55$\\\\

\section*{Exercise 2}
Solve $(x-3)^2 = 4$\\
\begin{align*}
    \sqrt{(x-3)^2} = \sqrt{4}\\
    x-3 \pm 2\\
    x - 3 = 2\\
    x = 5\\
    x - 3 = -2\\
    x = 1 \\
\end{align*}
Therefore $x = 5$ or $x = 1$

\section*{Exercise 3}
Solve $\sqrt{(x-2)} + 2 = 4$
\begin{align*}
    \sqrt{(x-2)} = 4 - 2\\
    \sqrt{(x-2)}^2 = 2^2\\
    x - 2 = 4\\
    x = 6\\
\end{align*}
Therefore $x = 6$
\end{document}
